\documentclass[a4j]{jarticle}

\usepackage[dvips,dvipdfmx]{graphicx}
%\usepackage{epsbox}
\usepackage{url}

\setlength{\headsep}{-5mm}
\setlength{\oddsidemargin}{0mm}
\setlength{\textwidth}{165mm}
\setlength{\textheight}{230mm}
\setlength{\footskip}{20mm}

\title{
\vspace{30mm}
{\bf 情報学群実験第3C レポート}
%{\bf 情報学群実験第4i レポート}
\\
\vspace{5mm}
第5回\\
\vspace{5mm}
       {\bf タイトル}
\vspace{90mm}
}

\author{
\vspace{5mm}
グループ3 \\
\vspace{5mm}
学籍番号 1180300 \\
\vspace{5mm}
{\large 大西 史洋}
\vspace{10mm}
}

%date{
%平成28年4月16日
%}

\begin{document}
\maketitle

\newpage
\begin{abstract} %600くらいまで

\end{abstract}

\section{目的}

\section{方法}
\subsection{開発環境}
今回使用した機器,ソフトウェアは下記のとおりである.

\begin{itemize}
  %------------------ハードウェア-----------------%
  \item ターミナル型パーソナルコンピュータ OptiPlex 3020\\
  メモリ 9.2GB\\
  プロセッサ Intel Xeon CPU e52667 v3(3.20GHz)\\
  開発 株式会社DELL

  %------------------ソフトウェア------------------%
  \item オペレーティングシステム Linux-215 バージョン6.7\\
  開発 Apple Inc.

  \item Webブラウザ FirefoxESR バージョン38.6.1\\
  開発 Mozilla Foundation

  \item 偽装用User Agent切り替えアドオン User Agent Switcher バージョン 0.7.3\\
  開発 Chris Pederick

  \item 画像処理ソフトウェア ImageJ バージョン 1.50\\
  開発 Wayne Rasband

  \item 色盲画像シュミレーション画像変換プラグイン Vischeck バージョン 1.1\\
  開発 TIny Eyes,inc.

\end{itemize}

\section{要素技術}

% 要素技術 (end)

\section{実験方法}

% 実験方法 (end)

\section{実験結果}

% section 実験結果 (end)

\section{考察} % (fold)

% section 考察 (end)

\section{結論}

% 結論 (end)

\section{関連語句}

% 関連語句 (end)

\bibliographystyle{jplain}
\begin{thebibliography}{1}

\bibitem{cud_population}
NPO法人 カラーユニバーサルデザイン機構.\\
\newblock 色弱・CUDのFAQ.\\
\newblock http://www.cudo.jp/resource/FAQ\#002, 2016年5月19日閲覧.

\bibitem{prere}
情報学群・情報学コース 学修の手引き p24~25.\\
\newblock 篠森 敬三,妻鳥 貴彦,栗原 徹,福本 昌弘,横山 和俊,島村 和典,岩田 誠,清水 明宏.\\
\newblock 高知工科大学 情報学群, 2016年4月1日.

\bibitem{html5lib}
  羽田野 太巳.
  \newblock Javascript ライブラリー.\\
  \newblock http://www.html5.jp/library/, 2016年5月22日閲覧.

\bibitem{html}
  猿橋 大.
  \newblock HTMLとは?.\\
  \newblock http://www.htmq.com/htmlkihon/001.shtml, 2016年5月24日閲覧.

\bibitem{gaborfilter2}
  城ノ戸 秀基.
  \newblock 顔の静止画像特徴と時系列画像特徴の併用による表情認識システムの構築.\\
  \newblock http://repo.lib.hosei.ac.jp/bitstream/10114/2687/1/15.pdf, 2016年5月15日閲覧.

\bibitem{eface}
  北原 章雄, 柳井, 啓司.
  \newblock Web画像ニュースの顔と人物名の対応付け.\\
  \newblock http://mm.cs.uec.ac.jp/kitaha-a/cvim159.pdf, 2016年5月15日閲覧.

\bibitem{templa}
  秋月  秀一, 櫻本 奏憲
  \newblock 基本的なテンプレートマッチング.\\
  \newblock http://isl.sist.chukyo-u.ac.jp/Archives/tm.html, 2016年5月17日閲覧
\end{thebibliography}
\newpage

\appendix

\section{主専攻講義一覧表の作成のHTMLファイル}

% ソースファイル (end)

\end{document}